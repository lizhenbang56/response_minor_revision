\documentclass[12pt]{article}
\usepackage{geometry}
\usepackage{times}
\usepackage{soul}
\usepackage{url}
\usepackage[hidelinks]{hyperref}
\usepackage[utf8]{inputenc}
\usepackage[small]{caption}
\usepackage{graphicx}
\usepackage{subfigure}
\usepackage{amsmath}
\usepackage{amsthm}
\usepackage{booktabs}
\usepackage{algorithm}
\usepackage{algorithmic}
\usepackage{lipsum}
\usepackage{multirow}  % for multirow command used in the table
\usepackage{setspace}
\usepackage{ulem}  % 用于添加下划线
\usepackage{enumitem}
\usepackage{color}
\usepackage{amssymb}  % 用于打对号
\newcommand{\eg}{e.g.}
\newcommand{\ie}{i.e.}
\urlstyle{same}

\newtheorem{example}{Example}
\newtheorem{theorem}{Theorem}

\hyphenpenalty=1
\tolerance=10000

\begin{document}
\setstretch{1.5}
\linespread{1}
\title{Response to Reviewers of TCSVT-07024-2021.R1: A Simple and Strong Baseline for Universal Targeted Attacks on Siamese Visual Tracking}
\author{Zhenbang Li, Yaya Shi, Jin Gao, Shaoru Wang, \\Bing Li, Pengpeng Liang, Weiming Hu}
\date{}
\maketitle

\noindent Dear Editors:

We would like to express our heartfelt gratitude to you and the reviewers for the insightful and helpful comments. When we revised the paper, we carefully considered and followed all the comments and suggestions provided by you and the reviewers. To summarize, we have made the following revisions:

(1) We have carefully considered and followed all the comments and suggestions related to the clarity of the writing, and made the new material a thoroughly revised manuscript. Specifically, we have performed a thorough spell and grammar check to increase the readability of our manuscript. In addition, we have placed the table XVI before references.

(2) We have re-organized the forms and contents in experiments. Specifically, we have removed the redundant experimental results of FAN in Section IV.D. We have also condensed the first paragraph of Section IV.G.

We hope that our revised manuscript is now appropriate for publication in IEEE Transactions on Circuits and Systems for Video Technology. Specific responses to all the comments of each reviewer are included in the rest of this document and highlighted using bold font after the comments of each reviewer for the convenience of cross-reference. To make the changes easier to identify where necessary, we also have underlined most of the revised parts in the manuscript and provide an underlined version for the convenience of second review.\\[10pt]
\indent We are looking forward to your reply.\\[10pt]
\noindent Yours sincerely,\\
\noindent Zhenbang Li, Yaya Shi, Jin Gao, Shaoru Wang, Bing Li, Pengpeng Liang, Weiming Hu
\\
\\
\\
\noindent Dr. Jin Gao (Contact author)\\
\noindent National Laboratory of Pattern Recognition (NLPR)\\
\noindent Institute of Automation, Chinese Academy of Sciences (CASIA)\\
\noindent Address: No. 95, Zhongguancun East Road, Haidian District,\\
\noindent Beijing 100190, P. R. China\\
\noindent Email: jin.gao@nlpr.ia.ac.cn

%%%%%%%%%%%%%%%%% 审稿人 1 %%%%%%%%%%%%%%%%%
\newpage
{\centering\section*{Response Letter to Reviewer \#1}}
\noindent Dear Reviewer \#1:

Thank you very much for your thorough review. Your insightful comments are very helpful for us to improve the quality of the paper. According to your comments and suggestions, we have carefully and extensively revised the manuscript. The main revised parts are highlighted by underlines in the underlined version for your convenience. You will find that all your comments and suggestions are considered and followed. We hope that our revised manuscript is now appropriate for publication in IEEE Transactions on Circuits and Systems for Video Technology.
In addition, point-to-point responses to your comments are given below and highlighted using bold font in line with your comments in order to facilitate cross-referencing.\\[10pt]
\indent We are looking forward to your reply.\\[10pt]
\noindent Yours sincerely,\\
\noindent Zhenbang Li, Yaya Shi, Jin Gao, Shaoru Wang, Bing Li, Pengpeng Liang, Weiming Hu
\\
\\
\\
\noindent Dr. Jin Gao (Contact author)\\
\noindent National Laboratory of Pattern Recognition (NLPR)\\
\noindent Institute of Automation, Chinese Academy of Sciences (CASIA)\\
\noindent Address: No. 95, Zhongguancun East Road, Haidian District,\\
\noindent Beijing 100190, P. R. China\\
\noindent Email: jin.gao@nlpr.ia.ac.cn

\newpage
\textit{good work, valued to be accepted as is.}

\textbf{Many thanks for your positive comments on the strength of our paper and the novelty of the proposed attack method. In addition, we have performed a thorough spell and grammar check to increase the readability of our manuscript. Specifically, we have replaced ``Thus it is necessary to perturb the template image to cooling down hot regions where the real target exists and increasing the responses at the position of the \textit{fake target}.'' with} ``\uline{Thus it is necessary to perturb the template image to cool down hot regions where the real target exists and increase the responses at the position of the \textit{fake target}.}''
\textbf{in Section III.A of the revised manuscript.}

\textbf{We have replaced ``FAN generates independent perturbation for each frame, while our perturbations are universal.'' with} ``\uline{FAN independently generates different perturbations for each frame, while our perturbations are universal.}'' \textbf{in Section IV.D of the revised manuscript.}

\textbf{We have replaced ``The main limitation of our work is that the translucent perturbations my result in suspicious attacks.'' with} ``\uline{The main limitation of our work is that the translucent perturbations may result in suspicious attacks}'' \textbf{in Section IV.G of the revised manuscript.}

\textbf{We have replaced ``Online learning has play an important role in correlation filter-based tracking, and the recent works (e.g., [17], [18], [19]) have largely advance the research in this area by exploring various online update strategies.'' with} ``\uline{Online learning has play an important role in correlation filter-based tracking, and the recent works (e.g., [17], [18], [19]) have largely advanced the research in this area by exploring various online update strategies.}'' \textbf{in Section II.A of the revised manuscript.}

\textbf{We have replaced ``Our method adds a \textit{small} universal patch to the search image to perform targeted attacks, which is is as translucent as in Baseline-1.'' with} ``\uline{Our method adds a \textit{small} universal patch to the search images to perform targeted attacks, which is as translucent as in Baseline-1.}'' \textbf{in Section IV.D of the revised manuscript.}

\textbf{We have replaced ``Our perturbations are trained using datasets with groundtruth box information.'' with} ``\uline{Our perturbations are trained using datasets with ground-truth box information.}'' \textbf{in Section IV.E of the revised manuscript.}

%%%%%%%%%%%%%%%%% 审稿人 2 %%%%%%%%%%%%%%%%%
\newpage
{\centering\section*{Response Letter to Reviewer \#2}}
\noindent Dear Reviewer \#2:

Thank you very much for your thorough review. Your insightful comments are very helpful for us to improve the quality of the paper. According to your comments and suggestions, we have carefully and extensively revised the manuscript. The main revised parts are highlighted by underlines in the underlined version for your convenience. You will find that all your comments and suggestions are considered and followed. We hope that our revised manuscript is now appropriate for publication in IEEE Transactions on Circuits and Systems for Video Technology.
In addition, point-to-point responses to your comments are given below and highlighted using bold font in line with your comments in order to facilitate cross-referencing.\\[10pt]
\indent We are looking forward to your reply.\\[10pt]
\noindent Yours sincerely,\\
\noindent Zhenbang Li, Yaya Shi, Jin Gao, Shaoru Wang, Bing Li, Pengpeng Liang, Weiming Hu
\\
\\
\\
\noindent Dr. Jin Gao (Contact author)\\
\noindent National Laboratory of Pattern Recognition (NLPR)\\
\noindent Institute of Automation, Chinese Academy of Sciences (CASIA)\\
\noindent Address: No. 95, Zhongguancun East Road, Haidian District,\\
\noindent Beijing 100190, P. R. China\\
\noindent Email: jin.gao@nlpr.ia.ac.cn

\newpage
\textit{The authors have now greatly improved understandability and clarity of their manuscript. Some additional experiments are also added to further illustrate the effectiveness of the proposed method.}

\textbf{Many thanks for your positive comments on the strength of our paper and the novelty of the proposed attack method.}

\textit{There are some minor comments: 1. Eq.3 and Eq.5 seem to be the same.}

\textbf{Sorry that in our previous manuscript we did not clarify the difference between Eq. (3) and Eq. (5). In Eq. (5), $p_\textbf{x}$ is a small (i.e, $64\times64$) patch added into the search image to achieve targeted attacks, while in Eq. (3), the perturbation $\delta_\textbf{x}$ is added to the whole search image $\textbf{x}$ and limited to untargeted attacks. We have analysed the difference between Eq. (3) and Eq. (5) in Section III.A of the revised manuscript. For your convenience of cross checking, the new added text is shown as follows:}

\uline{$A_{\text{add}}$ is a patch application operator which adds a small patch into the search image according to $b^{fake}_{\textbf{x}}$. This also departs from Eq. (3) in that the universal adversarial perturbation $\delta_\textbf{x}$ is added to the whole search image $\textbf{x}$ and limited to untargeted attacks.}

\textit{2. The section of experiments is a little tedious and some analyses in Sec. IV-D and Sec. IV-G are duplicated. The reviewer suggests that the authors re-organize these forms and contents in experiments.}

\textbf{Thanks for the good comment. As suggested, we have re-organized the forms and contents in experiments. Specifically, we have removed the redundant experimental results of FAN in Section IV.D as follows:} \uline{Moreover, our universal perturbations achieve better targeted attack performance than FAN on OTB2015 as shown in Section IV.G.} \textbf{We have also condensed the first paragraph of Section IV.G as follows:} \uline{We firstly compare our attack method with the recent state-of-the-art attack methods on OTB2015, including CSA [55], RTAA [54], SPARK [7], FAN [6] and TTP [5]. Please refer to Sec. II-C for more details about these methods. The untargeted attack results in Table XIV show that our targeted attack method can also achieve superior untargeted attack performance as well as the RTAA, SPARK and TTP attack methods and even better results than CSA and FAN. What’s more, the targeted attack results in Table XV also show that our achieved precision score with respect to the \textit{fake trajectory} is up to 0.795, which is significantly better than FAN and TTP. This further demonstrates our superior targeted attack performance when using the adversarial patch for attacks.}

\textit{3. Please check the format of the references. Some of them are incompatible with others.}

\textbf{Thanks for the good comment. As suggested, we have checked the format of the references.}

\textit{4. The reviewer suggests a thorough spell and grammar check that would increase the readability of this MS.}

\textit{- On page 5, 'to cooling down' is a grammar error in line 29.}

\textit{- On page 8, ' FAN generates independent perturbation for each frame' has a grammar error in line 33.}

\textit{- On page 12, 'my result in' is a typo in line 46.}

\textbf{Thanks for the good comment. As suggested, we have performed a thorough spell and grammar check to increase the readability of our manuscript. Specifically, we have replaced ``Thus it is necessary to perturb the template image to cooling down hot regions where the real target exists and increasing the responses at the position of the \textit{fake target}.'' with} ``\uline{Thus it is necessary to perturb the template image to cool down hot regions where the real target exists and increase the responses at the position of the \textit{fake target}.}''
\textbf{in Section III.A of the revised manuscript.}

\textbf{We have replaced ``FAN generates independent perturbation for each frame, while our perturbations are universal.'' with} ``\uline{FAN independently generates different perturbations for each frame, while our perturbations are universal.}'' \textbf{in Section IV.D of the revised manuscript.}

\textbf{We have replaced ``The main limitation of our work is that the translucent perturbations my result in suspicious attacks.'' with} ``\uline{The main limitation of our work is that the translucent perturbations may result in suspicious attacks}'' \textbf{in Section IV.G of the revised manuscript.}

\textbf{We have replaced ``Online learning has play an important role in correlation filter-based tracking, and the recent works (e.g., [17], [18], [19]) have largely advance the research in this area by exploring various online update strategies.'' with} ``\uline{Online learning has play an important role in correlation filter-based tracking, and the recent works (e.g., [17], [18], [19]) have largely advanced the research in this area by exploring various online update strategies.}'' \textbf{in Section II.A of the revised manuscript.}

\textbf{We have replaced ``Our method adds a \textit{small} universal patch to the search image to perform targeted attacks, which is is as translucent as in Baseline-1.'' with} ``\uline{Our method adds a \textit{small} universal patch to the search images to perform targeted attacks, which is as translucent as in Baseline-1.}'' \textbf{in Section IV.D of the revised manuscript.}

\textbf{We have replaced ``Our perturbations are trained using datasets with groundtruth box information.'' with} ``\uline{Our perturbations are trained using datasets with ground-truth box information.}'' \textbf{in Section IV.E of the revised manuscript.}

%%%%%%%%%%%%%%%%% 审稿人 3 %%%%%%%%%%%%%%%%%
\clearpage
\newpage
{\centering\section*{Response Letter to Reviewer \#3}}
\noindent Dear Reviewer \#3:

Thank you very much for your thorough review. Your insightful comments are very helpful for us to improve the quality of the paper. According to your comments and suggestions, we have carefully and extensively revised the manuscript. The main revised parts are highlighted by underlines in the underlined version for your convenience. You will find that all your comments and suggestions are considered and followed. We hope that our revised manuscript is now appropriate for publication in IEEE Transactions on Circuits and Systems for Video Technology.
In addition, point-to-point responses to your comments are given below and highlighted using bold font in line with your comments in order to facilitate cross-referencing.\\[10pt]
\indent We are looking forward to your reply.\\[10pt]
\noindent Yours sincerely,\\
\noindent Zhenbang Li, Yaya Shi, Jin Gao, Shaoru Wang, Bing Li, Pengpeng Liang, Weiming Hu
\\
\\
\\
\noindent Dr. Jin Gao (Contact author)\\
\noindent National Laboratory of Pattern Recognition (NLPR)\\
\noindent Institute of Automation, Chinese Academy of Sciences (CASIA)\\
\noindent Address: No. 95, Zhongguancun East Road, Haidian District,\\
\noindent Beijing 100190, P. R. China\\
\noindent Email: jin.gao@nlpr.ia.ac.cn

\newpage

%%%% 问题 3.1 %%%%
\textit{I have no other questions about this revised version except one format problem.}

\textbf{Many thanks for your positive comments on the strength of our paper and the novelty of the proposed attack method.}

\textit{It would be much better to place table XVI before references.}

\textbf{Thanks for the good comment. As suggested, we have placed the table XVI before references. In addition, we have performed a thorough spell and grammar check to increase the readability of our manuscript. Specifically, we have replaced ``Thus it is necessary to perturb the template image to cooling down hot regions where the real target exists and increasing the responses at the position of the \textit{fake target}.'' with} ``\uline{Thus it is necessary to perturb the template image to cool down hot regions where the real target exists and increase the responses at the position of the \textit{fake target}.}''
\textbf{in Section III.A of the revised manuscript.}

\textbf{We have replaced ``FAN generates independent perturbation for each frame, while our perturbations are universal.'' with} ``\uline{FAN independently generates different perturbations for each frame, while our perturbations are universal.}'' \textbf{in Section IV.D of the revised manuscript.}

\textbf{We have replaced ``The main limitation of our work is that the translucent perturbations my result in suspicious attacks.'' with} ``\uline{The main limitation of our work is that the translucent perturbations may result in suspicious attacks}'' \textbf{in Section IV.G of the revised manuscript.}

\textbf{We have replaced ``Online learning has play an important role in correlation filter-based tracking, and the recent works (e.g., [17], [18], [19]) have largely advance the research in this area by exploring various online update strategies.'' with} ``\uline{Online learning has play an important role in correlation filter-based tracking, and the recent works (e.g., [17], [18], [19]) have largely advanced the research in this area by exploring various online update strategies.}'' \textbf{in Section II.A of the revised manuscript.}

\textbf{We have replaced ``Our method adds a \textit{small} universal patch to the search image to perform targeted attacks, which is is as translucent as in Baseline-1.'' with} ``\uline{Our method adds a \textit{small} universal patch to the search images to perform targeted attacks, which is as translucent as in Baseline-1.}'' \textbf{in Section IV.D of the revised manuscript.}

\textbf{We have replaced ``Our perturbations are trained using datasets with groundtruth box information.'' with} ``\uline{Our perturbations are trained using datasets with ground-truth box information.}'' \textbf{in Section IV.E of the revised manuscript.}

%%%%%%%%%%%%%%%%% 审稿人 4 %%%%%%%%%%%%%%%%%
\clearpage
\newpage
{\centering\section*{Response Letter to Reviewer \#4}}
\noindent Dear Reviewer \#4:

Thank you very much for your thorough review. Your insightful comments are very helpful for us to improve the quality of the paper. According to your comments and suggestions, we have carefully and extensively revised the manuscript. The main revised parts are highlighted by underlines in the underlined version for your convenience. You will find that all your comments and suggestions are considered and followed. We hope that our revised manuscript is now appropriate for publication in IEEE Transactions on Circuits and Systems for Video Technology.
In addition, point-to-point responses to your comments are given below and highlighted using bold font in line with your comments in order to facilitate cross-referencing.\\[10pt]
\indent We are looking forward to your reply.\\[10pt]
\noindent Yours sincerely,\\
\noindent Zhenbang Li, Yaya Shi, Jin Gao, Shaoru Wang, Bing Li, Pengpeng Liang, Weiming Hu
\\
\\
\\
\noindent Dr. Jin Gao (Contact author)\\
\noindent National Laboratory of Pattern Recognition (NLPR)\\
\noindent Institute of Automation, Chinese Academy of Sciences (CASIA)\\
\noindent Address: No. 95, Zhongguancun East Road, Haidian District,\\
\noindent Beijing 100190, P. R. China\\
\noindent Email: jin.gao@nlpr.ia.ac.cn

\newpage
\textit{The authors have addressed my comments. The paper is ready for publication.}

\textbf{Many thanks for your positive comments on the strength of our paper and the novelty of the proposed attack method. In addition, we have performed a thorough spell and grammar check to increase the readability of our manuscript. Specifically, we have replaced ``Thus it is necessary to perturb the template image to cooling down hot regions where the real target exists and increasing the responses at the position of the \textit{fake target}.'' with} ``\uline{Thus it is necessary to perturb the template image to cool down hot regions where the real target exists and increase the responses at the position of the \textit{fake target}.}''
\textbf{in Section III.A of the revised manuscript.}

\textbf{We have replaced ``FAN generates independent perturbation for each frame, while our perturbations are universal.'' with} ``\uline{FAN independently generates different perturbations for each frame, while our perturbations are universal.}'' \textbf{in Section IV.D of the revised manuscript.}

\textbf{We have replaced ``The main limitation of our work is that the translucent perturbations my result in suspicious attacks.'' with} ``\uline{The main limitation of our work is that the translucent perturbations may result in suspicious attacks}'' \textbf{in Section IV.G of the revised manuscript.}

\textbf{We have replaced ``Online learning has play an important role in correlation filter-based tracking, and the recent works (e.g., [17], [18], [19]) have largely advance the research in this area by exploring various online update strategies.'' with} ``\uline{Online learning has play an important role in correlation filter-based tracking, and the recent works (e.g., [17], [18], [19]) have largely advanced the research in this area by exploring various online update strategies.}'' \textbf{in Section II.A of the revised manuscript.}

\textbf{We have replaced ``Our method adds a \textit{small} universal patch to the search image to perform targeted attacks, which is is as translucent as in Baseline-1.'' with} ``\uline{Our method adds a \textit{small} universal patch to the search images to perform targeted attacks, which is as translucent as in Baseline-1.}'' \textbf{in Section IV.D of the revised manuscript.}

\textbf{We have replaced ``Our perturbations are trained using datasets with groundtruth box information.'' with} ``\uline{Our perturbations are trained using datasets with ground-truth box information.}'' \textbf{in Section IV.E of the revised manuscript.}

\end{document}

